\chapter{摘\texorpdfstring{\quad}{}要}

	计算机体系结构在过去几十年的计算机发展历史中不断的演化更替,催生出了种类繁多的计算机指令集架构。现今存在着多种主流的计算机指令集架构,如x86、ARM、MIPS、OpenRISC等。迈入21世纪以来,简洁明快的RISC指令集开始成为微处理器设计的主流,而现存的RISC指令集如MIPS、SPARC、ARM等则或多或少存在着历史遗留的缺陷,为了解决这一问题,伯克利大学在2010年启动了第五代的RISC项目,设计了全新的RISC-V指令集架构,其特点是免费开源以及提供了许多现代化微处理器设计的技术特性,包括可选的扩展指令集、自定义指令功能以及用户与特权架构分离等。基于以上的特性,RISC-V微处理器正迅速占据着除桌面型微处理器以外的市场。与此同时,随着微处理器架构复杂程度的不断提升,对微处理器进行设计的工具与框架也面临着随之而来的效率与生产力的挑战,传统的集成电路设计工具与方法学如基于Verilog/SystemVerilog的瀑布设计模式开始变得捉襟见肘。随着基于敏捷硬件设计方法学设计工具如Chisel、SpinalHDL的兴起,基于高级编程语言并吸收了软件工程学方法的各类新兴硬件设计工具及框架开始逐步被学术界以及工业界所接受。本文将介绍基于RISC-V的开源面向嵌入式场景的DSP微处理器及其SoC及其设计,其实现了RV32GCP指令集架构,在性能上能够对标ARM专为嵌入式设备设计的微处理器Cortex-M3。该微处理器能够填补当前国内对于开源RISC-V DSP微处理器的空白。同时,本文还介绍用于构建该微处理器的硬件设计与验证框架PyHCL,PyHCL是第一款基于Python的RTL级硬件构造语言,相比于Chisel,PyHCL提供更为简单易用的用户接口,较短的学习曲线以及完整的Python特性支持。其效率要优于Chisel,且相比仅支持单元测试的Chisel,PyHCL具有完整的验证环境支持,包括单元测试、功能测试、差分测试等。本文由两大主要部分组成:一是基于RISC-V的DSP微处理器,二是敏捷硬件设计工具PyHCL,在本文最后也会阐述用于本微处理器验证过程中所使用的PyHCL验证方法,并介绍用于本文实现的微处理器验证所使用的基于多级验证环境的差分测试框架。

\keywordsCN{计算机体系结构;微处理器;RISC-V;敏捷硬件设计;Python}

\chapter{Abstract}

	Computer architecture has evolved over the past decades of computer development history. Therefore, a wide variety of computer instruction set architectures have been created in the past few decades. Today, there are several mainstream instruction set architectures, such as x86, ARM, MIPS, OpenRISC, and so on. Entering the 21st century, the soncise and simple RISC instruction set has become the mainstream of microprocessor design, while the existing RISC instruction sets such as MIPS, SPARC, and ARM have more or less defects left over from history. In order to solve this problem, Berkeley University launched the fifth-generation RISC project in 2010, designed a new RISC-V instruction set architecture, which is characterized by free open source and provides many technical features of modern microprocessor design, including optional extensions Instruction sets, custom instruction capabilities, and separation of user and privilege architectures. Based on the characeristics mentioned above, RISC-V microprocessors are rapidly occupying the market other than desktop microprocessors. At the same time, with the continuous improvement of the microprocessor architecture complexity, the tools and frameworks for designing microprocessors also face the challenges of efficiency and productivity. Traditional integrated circuit design tools and methodologies such as The waterfall design pattern based on Verilog/SystemVerilog is starting to become stretched. With the rise of design tools based on agile hardware design methodologies such as Chisel and SpinalHDL, various emerging hardware design tools and frameworks based on high-level programming languages ​​and absorbing software engineering methods have gradually been accepted by academia and industry. This article will introduce the design of a RISC-V open source DSP embedded microprocessor and SoC, it implements the RV32GCP instruction set, which can match the performance of the embedded microprocessor design by ARM: Cortex-M3. The microprocessor can fill the gap for open source RISC-V DSP microprocessors. At the same time, this paper also introduces the hardware design and verification framework PyHCL used to build the microprocessor. PyHCL is the first RTL-level hardware construction language based on Python. Compared with Chisel, PyHCL provides a simpler and easier-to-use user interface, short learning curve ,and full Python feature support. The efficiency of PyHCL is better than Chisel. Compared to Chisel, which only supports unit testing, PyHCL has complete verification environment support, including unit testing, functional testing, differential testing, etc. This paper consists of two main parts: one is the design of the DSP microprocessor based on RISC-V, and the other is the agile hardware design tool PyHCL. At the end of this paper, the PyHCL verification method used in the verification process of this microprocessor will be described. Also, we would introduce the differential test framework based on the multi-level verification environment used for the verification of the microprocessor.

\keywordsEN{Computer Architecture; Microprocessor; RISC-V; Agile Hardware design; Python}