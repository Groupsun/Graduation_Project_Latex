\chapter{致\texorpdfstring{\quad}{}谢}

三年的研究生生活在此即将划上句号,回首在华园的七年年时光,成功与挫折相伴,喜悦与伤感俱在。在七年的华园生活中,虽不能称得上别样的多姿多彩,但我还是能够说我度过了人生中最为充实,最为自由的时光。在此期间,家人的陪伴,老师的指导,同学的鼓励,都是我成长过程中不可或缺的一部分。
首先我要感谢我的导师赖晓铮老师,对我的毕业设计方面给予悉心的指导。老师不仅仅能够作为一个指导者的角色为我指引方向,还能作为一个合作者的角色经常与我讨论相关问题,并且能够让我自由的在喜欢的领域里尽情学习,更加激发了我浓厚的兴趣。同时,还感谢华园七年中我的所有任教老师,他们教给我许多的知识,为我打下了坚实的基础。同时,我还要感谢实验室的师兄师姐们,他们在毕业设计项目上对我帮助了许多。
然后我想感谢我的同学们,首先是我的舍友邓骏杰、王博和张德赋,我与他们相互陪伴的研究生学习生活,互相扶持走过了3年的时光。在我的研究生期间,赵森华,陈海康,范港平,赵伟杰等同学都带给了我快乐的时光,我也向他们学习了很多东西。我还要感谢我的好友邓楚钧以及吴卓桁同学,在我失落或者彷徨的时候给予我欢笑和鼓励。我衷心祝愿你们在未来能够脚踏实地,更进一步。
最后,我想感谢我的父母,他们始终是我最坚实的后盾,无论是生活还是学习上,他们都给我莫大的支持,感谢你们25年来对我的养育以及关爱,以及在我背后默默的付出。在未来的学习工作生活中,我希望我能够再接再厉,更近一步。

\begin{minipage}[t]{0.945\textwidth}%
	\begin{flushright}
		陈若晖\\
%		\today\\	% 自动时间
		2022年3月30日\\	%固定时间
		于华南理工大学
		\par\end{flushright}
\end{minipage}

